\documentclass[12pt, a4paper]{article}


\usepackage[margin=0.9in]{geometry}
\usepackage{hyperref}

\usepackage{hyperref}
\hypersetup{
	colorlinks,
	citecolor=black,
	filecolor=black,
	linkcolor=black,
	urlcolor=blue
}

\usepackage{listings}
\lstset{
	basicstyle=\ttfamily,
	columns=fullflexible,
	language=C,
	numbers=none,
	stepnumber=1,
	frame=none,
	breaklines=true,
	showstringspaces=false,
	postbreak=\mbox{\textcolor{red}{$\hookrightarrow$}\space},
	basicstyle=\footnotesize\ttfamily,
	keywordstyle=\bfseries\color{green},
	commentstyle=\itshape\color{red},
	identifierstyle=\color{blue},
	stringstyle=\color{black},
}

%opening
\title{\Large{Reference Manual -- AB Programming Language}}
\author{Ankit Pant -- 2018201035 \\ Bhavi Dhingra -- 2018201058}
\date{}

\begin{document}
	\maketitle
	\noindent\rule{\textwidth}{1pt}
	\newpage
	
	\begin{abstract}
		AB is a compiled programming language. It is a very simple programming language and supports only the basic data-types, conditionals, etc. This reference manual  explains the syntax of the AB programming language.
	\end{abstract}
	\newpage
	\tableofcontents
	\newpage
	\section{Introduction}
		AB is a simple programming language that supports very basic constructs. It is a compiled programming language. The following sections provide a brief description of the language.
		
		\subsection{Notation}
			The various formal definitions of the programming language are defined as:
			$$ Nonterminal \to terminal \,|\,  Nonterminal \,|\, \epsilon $$
			Where: 
			\begin{itemize}
				\item \textbf{Nonterminal} is the the part of the rule from which terminals or other rules can be derived.
				\item\textbf{terminal} is the part of the rule from which no more rules can be derived and is the fundamental unit of the programming language.
			\end{itemize}
		\textbf{Some important points:}
		\begin{itemize}
			\item Within the formal definition if something needs to be present literally, it is enclosed in quotation marks (i.e. `` ").
			\item When more than one rule can be derived from a Nonterminal, each rule is separated by a pipe symbol '$|$'. 
			\item The epsilon symbol ($\epsilon$) in a rule represents a null-string. \\
		\end{itemize}
	\section[Micro-Syntax]{Micro Syntax (Lexical Analysis)}
		This section describes the tokens which are the part of the input program. They can be further sub-divided as:
		\subsection{Identifiers}
			Identifiers are tokens that are used to name and hence identify the variables, functions, etc. in the program.  They are described by the following lexical definitions: \\ \\
			$ Letter \to [\_]^*[a-zA-Z][\_a-zA-Z0-9]^* $ \\
			$ Digit \to [0-9]$ \\
		
		\subsection{Keywords}
			Keywords are reserved words that cannot be used as identifiers. They hold a special meaning to the programming language. The following keywords have been defined in the programming language:  \\ \\
			 \textit{include, if, else, for, while, int, char, signed, unsigned, bool, return, void, string, print, println, input, true, false, main, and , or, not, define, break, continue}
			 
	 	\subsection{Literals}
	 		Literals are notations for constant values of some built-in types. They are described by the following lexical definitions: \\ \\
	 		$CharLiteral \to \,'<any \, ASCII \, character \, except \, newline \, or \, the \, quote>' $ \\
	 		$ StringLiteral \to  ``<any  number of \, ASCII \, characters \, except \, newline \, or \, the \, quote>" $ \\
	 		$ IntegerLiteral \to DecimalInteger $ \\
	 		$ DecimalInteger \to Digit^+ $ \\
	 		$ Fraction\to``." Digit^+ $ \\
	 		$ FloatLiteral \to IntegerLiteral \, Fraction $ 
	 		
	 	\subsection{Operators}
	 		Operators are used to performed various mathematical, relational, logical, etc. operations. They are described by the following lexical definitions: \\ \\
	 		$ LogicalOptr \to and \,|\, or \,|\, not $ \\
	 		$ RelationalOptr \to \, < \,|\, > \,|\, == \,|\, <= \,|\, >= \,|\, != $ \\
	 		$ ArithmeticOptr \to + \,|\, - \,|\, * \,|\, / \,|\, \% $ \\
	 		$ BitwiseOptr \to \, << \,|\, >> \,|\, \& \,|\, [ \,|\, ] \,|\, ~ $ \\
	 		$ AssignmentOptr \to \, = \,|\, += \,|\, *= \,|\,  /= \,|\, \%= \,|\, \&= \,|\, |= \,|\, <<= \,|\, >>= $ \\
	 		
	 	\subsection{Data Types}
	 		These are the data types built into the programming language. They are described by the following lexical definitions: \\ \\
	 		$ PrimitiveType \to void \,|\, bool  \,| \, char  \,| \, int  \,| \, unsigned \,|\, float  \,|\, long \,|\, long \, long  \,|\, unsigned \, long \, long$ \\
	 		%$ CompoundType\to PrimitiveType\,[ParenthesisExpr]  \\ ~~~~~~~~~~~~~~~~~~~~~~~~~\,| \, PrimitiveType\,[ParenthesisExpr] \,[ParenthesisExpr] $ \\ \\
	 		
	 	\subsection{Comments}
	 		Comments are statements in the input program that are ignored by the compiler. This programming language supports only single line comments as of now. They are described as: \\ \\
	 		$ Comment \,\to\, {//}<any \, character \, except \, newline> $  \\
	 		
	\section[Macro Syntax]{Macro Syntax (Grammatical analysis)}
		This section describes the various Grammatical rules using which the programming language has been defined. The following subsections give a brief description of the various grammatical rules used in the language and describes the syntax of the language.
		
			\subsection{Variables}
				Variables are memory locations. They are described as: \\ \\
				$ VarDec \to DataType \, IdentifierList $ \\
				
				Example:
				\begin{lstlisting}
	char x;
	int[10] arr;
	bool[3][3] mat;
				\end{lstlisting}
				
		
			\subsection{Expressions}
				Expressions form a part of the statements in in this programming language. They are described as: \\ \\
				$ Atom \to True \,|\, False \,|\, Identifier \, MultiDim \,|\, Literal \,|\, FuncCall $ \\ 
				%$ AtomList \to Atom  \,|\, Atom``," AtomList $ \\
				$ExprList \to \epsilon \,|\, `,' ExprList \,|\, Expr \, ExprList$\\ 
				$ Expr \to Atom \,|\, UnaryExpr \,|\, BinaryExpr  \,|\, TernaryExpr \,|\, ``("Expr``)"$ \\ 
				$ UnaryExpr \to Atom \,|\, ``-" Atom \,|\, ``+" Atom \,|\, ``not" Atom$\\ %\,|\, Identifier \, IndexOptr $ \\ 
				$ BinaryExpr \to Expr \,  Optr \, Expr  $ \\
				$ TernaryExpr \to ``("Expr \, ``?" \, Expr \, ``:" \, Expr``)" $ \\\\
				Example:
				\begin{lstlisting}
	x = 123;
	y = x+100;
	z = (x+y)*100;
	x < (y+z)
				\end{lstlisting}
				
			\subsection{Block}
				A block encloses a list of statements. It starts with a left curly bracket and ends with a right curly bracket ( $\left\lbrace \right\rbrace$ ) Any variable declared in a block has a scope within the block only. It is described as: \\ \\
				$ Block \to \left\lbrace StmtList \right\rbrace $ \\\\
				Example:
	\begin{lstlisting}
{
	x = 123;
	y = x+100;
	z = (x+y)*100;
	print("Hello World!");
}
	\end{lstlisting}
			
			\subsection{Functions}
				Functions are subroutines that are defined in the program and can be used to define operations that need to be performed repeatedly. The last statement inside a function should be a return statement. They can be defined as: \\\\
				$ FuncDef \to DataType \, FuncName``("ParameterList``)" \, Block $ \\
				$FuncName \to Indentifier \,|\, ``main"$\\
				$ParameterList \to \epsilon \,|\, `,'\,ParameterList \,|\, DataType \,`\&' Identifier \, ParameterList \\
				~~~~~~~~~~~~~~~~~~~~~~~~\,|\,  DataType \, Identifier \, ParameterList$ \\
				Example:
\begin{lstlisting}
int Square(int x)
{
	return (x*x);
}
\end{lstlisting}
				
			\subsection{Function Calls}
				Function calls can be used to transfers the execution flow to a function which has been defined previously. They can be defined as: \\\\
				$ FuncCall  \to Identifier \, ``(" ExprList ``)" $ \\\\
				Example:
\begin{lstlisting}
int x = 2;
int y = Square(x);
\end{lstlisting}				
				
			\subsection{Statements}
				Statements form the majority of the programs. They can be subdivided into simple statements and compound statements. Simple statements are comprised within a single logical line whereas compound statements contain a group of statements. They can be defined as:\\\\
				$ Delim \to ; $ \\ 
				$ StmtList \to \epsilon \,|\, CompoundStmt\, StmtList \,|\, SimpleStmt \, StmtList$\\
				$ SimpleStmt \to Expr \, Delim \,|\, PrintStmt \, Delim  \,|\, PrintlnStmt \, Delim  \\ ~~~~~~~~~~~~~~~~~~~~~ \,|\, ContinueStmt \, Delim \,|\, BreakStmt \, Delim\,|\, IncludeStmt \,|\, ReturnStmt \, Delim \\  ~~~~~~~~~~~~~~~~~\,|\, VarDec \, Delim \,|\, FuncCall \, Delim  \,|\,  DefineStmt \,|\, InputStmt Delim$ \\  
				$ PrintStmt \to ``print ("ExprList")" $ \\
				$ PrintlnStmt \to ``println ("ExprList")" $ \\ 
				$ InputStmt \to ``input ("Identifier\, MultiDim``)" $ \\ 
				%$InputVar \to Indentifier \,|\, Identifier \, IndexOptr$\\
				$ ReturnStmt \to ``return " \, Expr $ \\ 
				$ BreakStmt \to ``break" $ \\ 
				$ ContinueStmt \to ``continue" $ \\ 
				$ IncludeStmt \to ``\#include" Identifier$ \\ 
				$ DefineStmt \to ``\#define" Identifier \, Literal $ \\ 
				$ CompoundStmt \to IfStmt \,|\, ForStmt \,|\, WhileStmt \,|\, FuncDef$ \\ 
				$ IfStmt \to ``if ("Expr")" \, Block \, ElseStmt $ \\
				 $ElseStmt \to \epsilon \,|\, ``else" \, Block \,|\, ``else" \, IfStmt  $ \\ 
				$ ForStmt \to ``for ("Expr \, Delim  \, Expr \, Delim \, IncrementStmt``)" \, Block \\$
				 $IncrementStmt \to \epsilon\,|\,  Expr$  \\ 
				$ WhileStmt \to ``while ("Expr``)" \, Block $ \\ 
				
				\subsection{Identifiers}
				Identifiers are tokens that are used to name and hence identify the variables, functions, etc. in the program. 
				They can be defined on a macro level as: \\\\
				%$ Identifier \to (Letter \,|\,  {\_} ) \, (Letter \,|\,  {\_}  \,|\, Digit)^*  $ \\
				$IdentifierList \to Identifier, \,  IdentifierList  \,|\,  Identifier $ \\
				
				
				\subsection{Literals}
				Literals are notations for constant values of some built-in types. They can be defined on a macro level as: \\\\
				$ Literal \to CharLiteral \,|\, StringLiteral \,|\, IntegerLiteral \,|\, FloatLiteral $ \\ 
				
				\subsection{Operators}
				Operators are used to performed various mathematical, relational, logical, etc. operations. They can be defined on a macro level as: \\\\
				$ Optr \to LogicalOptr \,|\, RelationalOptr \,|\,  ArithmeticOptr \,|\, BitwiseOptr \,|\, AssignmentOptr $ \\ 
			%	$ IndexOptr \to [Expr] $ \\
				
				\subsection{Data Types}
				These are the data types built into the programming language. They are sub-divided into primitive  and compound data types.  They can be defined on a macro level as: \\\\
				$ DataType \to  PrimitiveType \, MultiDim $ \\ 
				$ MultiDim\to \epsilon \,|\, ``["Expr``]" \, MultiDim  $ \\\\
				
				\vspace{8mm}
				Examples:
\begin{lstlisting}
=> // Simple Statements (=> indicates different statements)
=> print("Hello World!"); 
=> println("This will add a newline character at the end");
=> int x;
=> input(x);
=> // Compound Statements (=> indicates different statements)
=>if(x<100)
{
	println("Less than Hundred!");
}
=>else
{
	println("More than hundred");
}
\end{lstlisting} 			
\vspace{5mm}

	\section{Few Semantic Checks}
		The five semantic checks that may be done on an input program are:
		\begin{enumerate}
			\item \textbf{Type Checking:} It can be done to ensure that all operands in any expression are of appropriate types.
			\item \textbf{Scope Checking:} It can be done to constraint  the  visibility  of  an  identifier  to  some  subsection  of  the  program. 
			\item \textbf{Undeclared Variable Check:} It can be done to ensure that the variable has been declared before use.
			\item \textbf{Multiple Variable Declaration Check:} It can be done to ensure that the same variable has not been declared multiple times within a scope.
			\item \textbf{Function Call Arguments Check}: It can be done to ensure that the number of arguments as well as the type of arguments in the function calls match the function signatures. \\
		\end{enumerate}
	
	\section{Sample Programs}
	
		\subsection[Program 1]{A program which computes $ g(N,k) = \Sigma _{i=1} ^{N} \,i^k$ where N and k are given as inputs.}
\begin{lstlisting}
int main()
{
	int N;
	int k;
	int sum;
	sum = 0;
	print("Enter the value of N: ");
	input(N);
	print("Enter the value of k: ");
	input(k);
	int i;
	for(i=1;i<=N;i+=1) 
	{
		int temp = i;
		int j;
		for(j=2;j<=k;j+=1) 
		{
			temp = temp * i;
		}
		sum = sum + temp;
	} 
	print("The value of the expression = ");
	println(sum)
	return 0;
}
\end{lstlisting}

		\vspace{5mm}
		\subsection[Program 2]{A program to check if a given input number N is prime.}
\begin{lstlisting}
bool is_prime(int n)
{
	int i;
	for (i = 2; i*i <= n; i += 1)
	{
		if (n % i == 0)
		{
			return false;
		}
	}
	return true;
}

int main()
{
	int n;
	input(n);
	
	print (n);
	if (is_prime(n))
	{
		println (" is a prime number");
	}
	else
	{
		println (" is not a prime number");
	}
	return 0;
}

\end{lstlisting}

		\vspace{5mm}
		\subsection[Program 3]{A program Find the sum of all prime numbers strictly less than N where N is provided as an input.}
\begin{lstlisting}
int main()
{
	int N;
	print("Enter the value of N: ");
	input(N);
	bool[N] parr;
	parr[0] = false;
	parr[0] = false;
	int i;
	for(i=2;i<N;i+=1) 
	{
		parr[i] = true;
	}
	for(i=2;i<N;i+=1) 
	{
		if(parr[i]==true){
			int j;
			for(j=2*i;j<N;j+=i) 
			{
				parr[j] = false;
			}
		}
	}
	int sum;
	sum = 0;
	for(i=2;i<N;i+=1)
	{
		if(parr[i]==true){
			sum = sum + i;
		}
	}
	print("The sum of all primes strictly less than N = ");
	println(sum);
	return 0;
}
\end{lstlisting}		
			
		\vspace{5mm}
		\subsection[Program 4]{A program Enumerate all the Pythagorean triplets $ (x,y,z) $ where $ x,y,z $ are integers and $ x^2 + y^2 = z^2$ and $ z< 10000000 $. Output the count at the end.}
\begin{lstlisting}
int main()
{
	unsigned long long a, b, c, m, n;
	
	int d;
	for (d = 1; d < 100000000/2; d += 1)
	{
		for (n = 1; n < 100000000/2; n += 1)
		{
			m = n+d;
			c = m*m + n*n;
			if (c > 100000000)
			{
				break;
			}
			a = m*m - n*n;
			b = 2*m*n;
			println ("(", a, ",", b, ",", c, ")");
		}
	}
	return 0;
}

\end{lstlisting}

		\vspace{5mm}
		\subsection[Program 5]{A program to print all combinations of $ \left\lbrace 1, \ldots , n \right\rbrace $ where $n$ is given as an input.}
\begin{lstlisting}
void Combinations(int arr, int n, int l, int r)
{
	if(l==r){
		int i;
		for(i=0;i<n;i+=1)
		{
			print(arr[i]);
			print(",")
		}
		print(" ");
	}
	else{
		int i;
		for(i=l;i<=r;i+=1) 
		{
			int  temp;
			temp = arr[l];
			arr[l] = arr[i];
			arr[i] = temp;
			Combinations(arr,n,l+1,r);
			temp = arr[l];
			arr[l] = arr[i];
			arr[i] = temp;
		}
	}
}

int main()
{
	int N;
	print("Enter the value of N: ");
	input(N);
	int[N] arr;
	int i;
	for(i=0;i<N;i+=1)
	{
		arr[i] = (i+1);
	}
	println("The combinations are: ");
	Combinations(arr,N,0,N-1);
	println();
	return 0;
}
\end{lstlisting}

		\vspace{5mm}
		\subsection[Program 6]{A program for insertion sort.}
\begin{lstlisting}
void swap(int& a, int& b)
{
	int temp = a;
	a = b;
	b = temp;
}

int main()
{
	int n;
	print ("Enter the size of array: ");
	input (n);
	
	int[n] arr;
	println ("Enter the values:");
	int i;
	for (i = 0; i < n; i += 1)
	{
		input (arr[i]);
	}
	
	int i, j;
	for (j = 1; j < n; j += 1)
	{
		for (i = j-1; ()i >= 0) && ()arr[i] > arr[i+1]); i -= 1)
		{
			swap(arr[i], arr[i+1]);
		}
	}
	println ("After Insertion Sort:");
	for (i = 0; i < n; i += 1)
	{
		print (arr[i],);
	}
	println ();
	return 0;
}

\end{lstlisting}

		\vspace{5mm}
		\subsection[Program 7]{A program for Radix sort.}
\begin{lstlisting}
int n;

void Count_Sort(int[] arr, int index)
{
	int[n] aux_arr;
	int[10] count;
	int i;
	for(i=0;i<10;i+=1) 
	{
		count[i] = 0;
	}  
	for(i=0;i<n;i+=1)
	{
		int temp = (arr[i]/index)%10;
		count[temp] = count[temp]+1;
	}
	for(int i=1;i<10;i+=1) 
	{
		count[i] = count[i] + count[i-1];
	}
	for(i=n-1;i>=0;i-=1)
	{
		int temp = (arr[i]/index)%10;
		temp = temp - 1;
		int temp2 = count[temp];
		aux_arr[temp2] = arr[i];
		count[temp] = count[temp] - 1;
	}
	for(i=0;i<n;i+=1)
	{
		arr[i] = aux_arr[i];
	}
}

int main()
{
	print("Enter the number of elements in the array");
	input(n);
	int[n] arr;
	print("Enter the elements of the array")
	int i;
	for(i=0;i<n;i+=1)
	{
		input(arr[i]);
	}
	int maxe = arr[0];
	for(i=1;i<n;i+=1)
	{
		if(maxe>arr[i]){
			maxe = arr[i];
		}
	}
	int digits = 0;
	while(maxe>0)
	{
		digits = digits + 1;
		maxe  =maxe/10;
	}
	int exp = 1;
	for(i=1;i<=digits;i+=1)
	{
		Count_Sort(arr,exp);
		exp = exp * 10;
	}
	print("The sorted array after applying Radix Sort is");
	for(i=0;i<n;i+=1)
	{
		print(arr[i]);
		print(" ");
	}
	println();
	return 0;
}
\end{lstlisting}

		\vspace{5mm}
		\subsection[Program 8]{A program for Merge sort.}
\begin{lstlisting}
#define INT_MAX 1000000007

void merge(int[]& arr, int i, int m, int j)
{
	int arr1[m-i+1 + 1], arr2[j-m + 1];
	arr1[m-i+1] = arr2[j-m] = INT_MAX;
	
	int k;
	for (k = 0; k < m-i+1; k += 1)
	{
		arr1[k] = arr[i+k];
	}
	for (k = 0; k < j-m; k += 1)
	{
		arr2[k] = arr[m+k+1];
	}
	int p = 0, q = 0;
	for (k = i; k <= j; k += 1)
	{
		if (arr1[p] < arr2[q])
		{
			arr[k] = arr1[p];
			p += 1;
		}
		else
		{
			arr[k] = arr2[q];
			q += 1;
		}
	}
}

void merge_sort(int[]& arr, int n, int i, int j)
{
	if (i < j)
	{
		int m = i + (j-i)/2;
		merge_sort(arr, n, i, m);
		merge_sort(arr, n, m+1, j);
		merge(arr, i, m, j);
	}
}

int main()
{
	int n;
	print ("Enter the size of array: ");
	input (n);
	
	int arr[n];
	println ("Enter the values:");
	int i;
	for (i = 0; i < n; i += 1)
	{
		input (arr[i]);
	}
	
	merge_sort(arr, n, 0, n-1);
	println ("After Merge Sort: ");
	for (i = 0; i < n; i += 1)
	{
		print (arr[i],);
	}
	println ();
	return 0;
}

\end{lstlisting}

		\vspace{5mm}
		\subsection[Program 9]{A program to compute the sum of two input matrices.}
\begin{lstlisting}
int main()
{
	int n;
	int m;
	print("Enter the number of rows of the matrices: ");
	input(m)
	print("Enter the number of columns of the matrices: ");
	input(n)
	int[m][n] mat1;
	int[m][n] mat2;
	print("Enter the elements of first matrix");
	int i;
	int j;
	for(i=0;i<m;i+=1)
		{
		for(j=0;j<n;j+=1)
		{
			input(mat1[i][j]);
		}
	}
	print("Enter the elements of second matrix");
	for(i=0;i<m;i+=1) 
	{
		for(j=0;j<n;j+=1)
		{
			input(mat2[i][j]);
		}
	}
	int[m][n] sum_mat;
	for(i=0;i<m;i+=1) 
	{
		for(j=0;j<n;j+=1) 
		{
			sum_mat[i][j] = mat1[i][j] + mat2[i][j];
		}
	}
	print("The sum of the matrices is");
	for(i=0;i<m;i+=1)
	{
		for(j=0;j<n;j+=1) 
		{
			print(sum_mat[i][j])
			print(" ");
	}
		println();
	}
	return 0;
}
\end{lstlisting}

		\vspace{5mm}
		\subsection[Program 10]{A program compute the product of two input matrices.}
\begin{lstlisting}
int main ()
{
	int A_m, A_n;
	print ("Enter dimensions of matrix A: ");
	input (A_m)
	input (A_n);
	int[A_m][A_n] A;
	println ("Enter values of matrix A:");
	int i, j;
	for (i = 0; i < A_m; i += 1)
	{
		for (j = 0; j < A_n; j += 1)
		{
			input (A[i][j]);
		}
	}
	
	int B_m, B_n;
	println ();
	print ("Enter dimensions of matrix B: ");
	input (B_m);
	input (B_n);
	int[B_m][B_n] B;
	println ("Enter values of matrix B:");
	for (i = 0; i < B_m; i += 1)
	{
		for (j = 0; j < B_n; j += 1)
		{
			input (B[i][j]);
		}
	}
	
	if (A_n != B_m)
	{
		println ("The matrices are not compatible for multiplication!!");
		return 1;
	}
	
	int[A_m][B_n] C;
	for (i = 0; i < A_m; i += 1)
	{
		for (j = 0; j < B_n; j += 1)
		{
			int k;
			C[i][j] = 0;
			for (int k = 0; k < A_n; k += 1)
			{
				C[i][j] += A[i][k] * B[k][j];
			}
		}
	}
	
	println();
	println ("A x B: ");
	for (int i = 0; i < A_m; ++i)
	{
		for (int j = 0; j < B_n; ++j)
		{
			print (C[i][j],);
		}
		println ();
	}
	return 0;
}

\end{lstlisting}			
				
\end{document}
